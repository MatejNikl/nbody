%% Technical report, BI-EIA
%% Authors: Jan Bobek, Matej Nikl
%%

\documentclass[10pt,a4paper]{article}

\usepackage[czech]{babel}
\usepackage[utf8]{inputenc}
\usepackage{fullpage}
\usepackage{amsmath}
\usepackage{datetime}
\usepackage{listings}
\usepackage{color}
\usepackage{graphicx}

\title{Modelování částic \\ Technická zpráva z předmětu BI-EIA}
\author{Jan Bobek, Matěj Nikl}
\date{\today}
\lstset{language=C++, numbers=left}

\begin{document}
\maketitle

\section{Úvod}

\subsection{Popis problému}
Uvažujme $ N $ částic $ A_1\ldots A_N $ ve dvourozměrném prostoru působící
na sebe navzájem gravitační a elektrostatickou silou. Podle Newtonova
gravitačního zákona se částice $ A_i $ a $ A_j $ přitahují silou $ \vec{F_g} $
o velikosti (při zanedbání konstant):

$$ |\vec{F_g}| = \frac{m_i m_j}{|\vec{A_i A_j}|^2} $$

kde $ m_i $ a $ m_j $ jsou hmotnosti jednotlivých částic a
$ |\vec{A_i A_j}| $ jejich vzdálenost. Jelikož gravitační síla je vždy
přitažlivá, směr vektoru $ \vec{F_g} $ je shodný se směrem vektoru
$ \vec{A_i A_j} $, resp. $ \vec{A_j A_i} $.

Obdobně můžeme formulovat druhou uvažovanou, tj. elektrostatickou sílu.
Podle Coulombova zákona se částice $ A_i $ a $ A_j $ přitahují, resp. odpuzují
silou $ F_q $ o velikosti (opět zanedbáváme konstanty):

$$ |\vec{F_q}| = \frac{q_i q_j}{|\vec{A_i A_j}|^2} $$

kde $ q_i $ a $ q_j $ jsou náboje obou částic a $ |\vec{A_i A_j}| $
opět jejich vzdálenost. Na rozdíl od gravitační síly je však elektrostatická
síla přitažlivá při rozdílu znamének nábojů a odpudivá při jejich shodě. Směr
vektoru $ F_q $ je tedy shodný se směrem vektoru $ \pm\vec{A_i A_j} $, resp.
$ \pm\vec{A_j A_i} $.

Pokusme se nyní přímo vyjádřit vektor $ \vec{F_g} $, resp. $ \vec{F_q} $.
Víme, že $ \vec{F_g} $ má stejný směr jako $ \vec{A_i A_j} $, tedy:

$$ \vec{F_g} = k_g\cdot\vec{A_i A_j} \hspace{20pt} k_g > 0 $$

Po dosazení do rovnice uvedené výše a několika drobných úpravách dostáváme:

\begin{align*}
|k_g\cdot\vec{A_i A_j}| &= \frac{m_i m_j}{|\vec{A_i A_j}|^2} \\
k_g\cdot|\vec{A_i A_j}| &= \frac{m_i m_j}{|\vec{A_i A_j}|^2} \\
k_g &= \frac{m_i m_j}{|\vec{A_i A_j}|^3} \\
\vec{F_g} &= \frac{m_i m_j}{|\vec{A_i A_j}|^3} \cdot \vec{A_i A_j}
\end{align*}

Obdobně můžeme psát pro $ F_q $:

\begin{align*}
k_q\cdot\vec{A_i A_j} &= \vec{F_q} \\
|k_q\cdot\vec{A_i A_j}| &= \frac{q_i q_j}{|\vec{A_i A_j}|^2}
\end{align*}

Zde se na chvíli pozastavme. Je-li znaménko $ q_i $ a $ q_j $ shodné,
tedy je-li jejich součin větší než nula, pak se částice odpuzují a
$ k_q < 0 $. Naopak, je-li jejich součin menší než nula, pak se částice
přitahují a $ k_q > 0 $. Můžeme tedy psát, že:

\begin{align*}
k_q\cdot|\vec{A_i A_j}| &= -\frac{q_i q_j}{|\vec{A_i A_j}|^2} \\
k_q &= -\frac{q_i q_j}{|\vec{A_i A_j}|^3} \\
\vec{F_q} &= -\frac{q_i q_j}{|\vec{A_i A_j}|^3} \cdot \vec{A_i A_j}
\end{align*}

Uvažujme nyní součet všech sil působící na částici $ A_i $. Taková síla
$ \vec{F_i} $ je rovna:

\begin{align*}
\vec{F_i} &= \vec{F_g} + \vec{F_q} \\
 &= \sum_{j = 1, i \neq j}^{N}{\frac{m_i m_j}{|\vec{A_i A_j}|^3} \cdot \vec{A_i A_j}}
  + \sum_{j = 1, i \neq j}^{N}{-\frac{q_i q_j}{|\vec{A_i A_j}|^3} \cdot \vec{A_i A_j}} \\
 &= \sum_{j = 1, i \neq j}^{N}{\frac{m_i m_j - q_i q_j}{|\vec{A_i A_j}|^3} \cdot \vec{A_i A_j}}
\end{align*}

Podle Newtonova druhého gravitačního zákona ($ \vec{F_i} = m_i\vec{a_i} $) pak
snadno odvodíme zrychlení $ a_i $ částice $ A_i $:

\begin{align*}
m_i\vec{a_i} &=
  \sum_{j = 1, i \neq j}^{N}{\frac{m_i m_j - q_i q_j}{|\vec{A_i A_j}|^3} \cdot \vec{A_i A_j}} \\
\vec{a_i} &=
  \sum_{j = 1, i \neq j}^{N}{\frac{m_j - \frac{q_i q_j}{m_i}}{|\vec{A_i A_j}|^3} \cdot \vec{A_i A_j}}
\end{align*}

Nyní si zvolme jistou (dostatečně malou) časovou konstantu $ dt $
reprezentující přesnost simulace pohybu částic v prostoru. Zjistíme, že za
časový okamžik $ dt $ částice $ A_i $ změní svoji polohu a rychlost
následujícím způsobem:

\begin{align*}
A_i' &= A_i + \vec{v_i} dt + \frac{1}{2}\vec{a_i} dt^2 \\
\vec{v_i'} &= \vec{v_i} + \vec{a_i} dt
\end{align*}

Poslední tři uvedené rovnice tvoří teoretický základ našeho algoritmu.

\subsection{Sekvenční algoritmus}

\begin{lstlisting}
for (step = 0; step < s.n_steps && !g_interrupted; ++step) {
    for (unsigned int i = 0; i < s.n_particles; ++i) {
        float ax = 0.0f;
        float ay = 0.0f;

        for (unsigned int j = 0; j < s.n_particles; ++j) {
            float dx = x[j] - x[i];
            float dy = y[j] - y[i];
            float invr = 1.0f / std::sqrt(dx * dx + dy * dy + 0.5f);
            float coef = (m[j] - q[i] * q[j] / m[i]) * invr * invr * invr;

            ax += coef * dx;
            ay += coef * dy;
        }

        xn[i] = x[i] + vx[i] * dt + 0.5f * ax * dt * dt;
        yn[i] = y[i] + vy[i] * dt + 0.5f * ay * dt * dt;
        vx[i] += ax * dt;
        vy[i] += ay * dt;
    }
}
\end{lstlisting}

Z výpisu kódu můžeme vidět, že jádro programu není dlouhé ani složité. Jedná
se víceméně o přímý přepis matematických rovnic uvedených výše až na několik
implementačních detailů.

Na řádku 9 si můžete všimnout, že před výpočtem odmocniny je k vstupní
hodnotě připočtena konstanta $ 0.5 $. Jedná se o protiopatření pro případ,
že $ i = j $; bez konstanty bychom se snažili odmocnit nulu a následující
dělení by selhalo pro dělení nulou. Nepřesnost zanesená touto konstantou
je zanedbatelná a jedná se výkonnostně o lepší řešení než tento případ
explicitně ošetřovat.

Na 10. řádku je zase zřejmé, že raději než třikrát dělit vzdáleností částic
je lepší si spočítat její převrácenou hodnotu a tou třikrát násobit. Výhodnost
spočívá v pozorování, že je obvykle časově náročnější provádět dělení než
násobení.

Poslední věcí stojící za zmínku je časová složitost algoritmu, která
je z výpisu kódu zcela zjevně $ O(n^2) $.

\section{Optimalizovaná verze}

Hlavní provedenou optimalizací bylo čtyřnásobné rozbalení smyčky cyklící přes
proměnnou \texttt{i} a následná vektorizace. Proměnné \texttt{ax}, \texttt{ay},
\texttt{dx}, \texttt{dy}, \texttt{invr} a \texttt{coef} byly modifikovány na
vektorové proměnné, přístup do polí \texttt{x}, \texttt{y}, \texttt{xn},
\texttt{yn}, \texttt{m}, \texttt{q}, \texttt{vx} a \texttt{vy} byl též
vektorizován. Pro výpočet odmocniny byla ručně použita SSE instrukce
\texttt{rsqrtps}.

Také jsme zjistili, že volba optimalizací \texttt{-Ofast} zvyšuje oproti
\texttt{-O3} výkon obou verzí algoritmu, avšak zejména té neoptimalizované.
Podrobné výsledky viz níže.

\begin{figure}[h]
    \centering
    \includegraphics[width=0.91\textwidth]{graph1.eps}
    %%\caption{}
    \label{fig:1}
\end{figure}

Z měření je dále vidět, že ani pro nejvyšší měřitelné $ n = 440000 $, kdy doba
běhu naivní verze již atakovala limitních 60 minut, nedochází k žádnému
viditelnému zpomalení z důvodu výpadků cache. Dalo by se tedy předpokládat,
že loop tiling nepřinese žádné zrychlení $ - $ což se také na naší
experimentální verzi potvrdilo $ - $ verze s loop tilingem byla vždy méně
výkonná, a proto jsme se rozhodli se jí nijak dále nezabývat.

\section{Paralelizovaná verze}

Po počáteční úvaze jsme se rozhodli paralelizovat pouze smyčku cyklící
přes proměnnou \texttt{i}, vzhledem k jednoduché predikci chování
paralelizovaného algoritmu a primitivnímu plánování práce na vlákna, tj.
minimální režii. Každá iterace představuje stejný díl práce a jednotlivé
iterace tohoto cyklu mohou běžet plně paralelně bez jakýchkoliv nároků na
synchronizaci (pokud bychom paralelizovali i cyklus proměnné \texttt{j},
bylo by nutné synchronizovat proměnné \texttt{ax} a \texttt{ay}).

\begin{lstlisting}
for (step = 0; step < s.n_steps && !g_interrupted; ++step) {
#   pragma omp parallel for schedule(static) default(none) \
        firstprivate(x, y, xn, yn, vx, vy, m, q) shared(s)
    for (unsigned int i = 0; i < s.n_particles - 3; i += 4) {
\end{lstlisting}

Direktiva odpovídá charakteristice popsané v předchozím odstavci, využíváme
statického plánování práce a drtivá většina proměnných je privátních. Jedná
se totiž o ukazatele, které nejsou v cyklu měněny a je z nich buď pouze čteno
a nebo pouze zapisováno na disjuktní paměťová místa. Jedinou výjimkou je
proměnná \texttt{s}, která je typu reference na třídu a jako taková nemůže
být privátní. Naštěstí je používána pouze pro kontrolu okrajových podmínek
smyček proměnných \texttt{i} a \texttt{j}.

\section{Závěr}

\end{document}

