%% Technical report, BI-EIA
%% Authors: Jan Bobek, Matej Nikl
%%

\documentclass[10pt,a4paper]{article}

\usepackage[czech]{babel}
\usepackage[utf8]{inputenc}
\usepackage{fullpage}
\usepackage{amsmath}
\usepackage{datetime}
\usepackage{listings}
\usepackage{color}
\usepackage{graphicx}

\title{Modelování částic \\ Technická zpráva z předmětu BI-EIA}
\author{Jan Bobek, Matěj Nikl}
\date{\today}
\lstset{language=C++, numbers=left}

\begin{document}
\maketitle

\section{Úvod}

\subsection{Popis problému}
Uvažujme $ N $ částic $ A_1\ldots A_N $ ve dvourozměrném prostoru působící
na sebe navzájem gravitační a elektrostatickou silou. Podle Newtonova
gravitačního zákona se částice $ A_i $ a $ A_j $ přitahují silou $ \vec{F_g} $
o velikosti (při zanedbání konstant):

$$ |\vec{F_g}| = \frac{m_i m_j}{|\vec{A_i A_j}|^2} $$

kde $ m_i $ a $ m_j $ jsou hmotnosti jednotlivých částic a
$ |\vec{A_i A_j}| $ jejich vzdálenost. Jelikož gravitační síla je vždy
přitažlivá, směr vektoru $ \vec{F_g} $ je shodný se směrem vektoru
$ \vec{A_i A_j} $, resp. $ \vec{A_j A_i} $.

Obdobně můžeme formulovat druhou uvažovanou, tj. elektrostatickou sílu.
Podle Coulombova zákona se částice $ A_i $ a $ A_j $ přitahují, resp. odpuzují
silou $ F_q $ o velikosti (opět zanedbáváme konstanty):

$$ |\vec{F_q}| = \frac{q_i q_j}{|\vec{A_i A_j}|^2} $$

kde $ q_i $ a $ q_j $ jsou náboje obou částic a $ |\vec{A_i A_j}| $
opět jejich vzdálenost. Na rozdíl od gravitační síly je však elektrostatická
síla přitažlivá při rozdílu znamének nábojů a odpudivá při jejich shodě. Směr
vektoru $ F_q $ je tedy shodný se směrem vektoru $ \pm\vec{A_i A_j} $, resp.
$ \pm\vec{A_j A_i} $.

Pokusme se nyní přímo vyjádřit vektor $ \vec{F_g} $, resp. $ \vec{F_q} $.
Víme, že $ \vec{F_g} $ má stejný směr jako $ \vec{A_i A_j} $, tedy:

$$ \vec{F_g} = k_g\cdot\vec{A_i A_j} \hspace{20pt} k_g > 0 $$

Po dosazení do rovnice uvedené výše a několika drobných úpravách dostáváme:

\begin{align*}
|k_g\cdot\vec{A_i A_j}| &= \frac{m_i m_j}{|\vec{A_i A_j}|^2} \\
k_g\cdot|\vec{A_i A_j}| &= \frac{m_i m_j}{|\vec{A_i A_j}|^2} \\
k_g &= \frac{m_i m_j}{|\vec{A_i A_j}|^3} \\
\vec{F_g} &= \frac{m_i m_j}{|\vec{A_i A_j}|^3} \cdot \vec{A_i A_j}
\end{align*}

Obdobně můžeme psát pro $ F_q $:

\begin{align*}
k_q\cdot\vec{A_i A_j} &= \vec{F_q} \\
|k_q\cdot\vec{A_i A_j}| &= \frac{q_i q_j}{|\vec{A_i A_j}|^2}
\end{align*}

Zde se na chvíli pozastavme. Je-li znaménko $ q_i $ a $ q_j $ shodné,
tedy je-li jejich součin větší než nula, pak se částice odpuzují a
$ k_q < 0 $. Naopak, je-li jejich součin menší než nula, pak se částice
přitahují a $ k_q > 0 $. Můžeme tedy psát, že:

\begin{align*}
k_q\cdot|\vec{A_i A_j}| &= -\frac{q_i q_j}{|\vec{A_i A_j}|^2} \\
k_q &= -\frac{q_i q_j}{|\vec{A_i A_j}|^3} \\
\vec{F_q} &= -\frac{q_i q_j}{|\vec{A_i A_j}|^3} \cdot \vec{A_i A_j}
\end{align*}

Uvažujme nyní součet všech sil působící na částici $ A_i $. Taková síla
$ \vec{F_i} $ je rovna:

\begin{align*}
\vec{F_i} &= \vec{F_g} + \vec{F_q} \\
 &= \sum_{j = 1, i \neq j}^{N}{\frac{m_i m_j}{|\vec{A_i A_j}|^3} \cdot \vec{A_i A_j}}
  + \sum_{j = 1, i \neq j}^{N}{-\frac{q_i q_j}{|\vec{A_i A_j}|^3} \cdot \vec{A_i A_j}} \\
 &= \sum_{j = 1, i \neq j}^{N}{\frac{m_i m_j - q_i q_j}{|\vec{A_i A_j}|^3} \cdot \vec{A_i A_j}}
\end{align*}

Podle Newtonova druhého gravitačního zákona ($ \vec{F_i} = m_i\vec{a_i} $) pak
snadno odvodíme zrychlení $ a_i $ částice $ A_i $:

\begin{align*}
m_i\vec{a_i} &=
  \sum_{j = 1, i \neq j}^{N}{\frac{m_i m_j - q_i q_j}{|\vec{A_i A_j}|^3} \cdot \vec{A_i A_j}} \\
\vec{a_i} &=
  \sum_{j = 1, i \neq j}^{N}{\frac{m_j - \frac{q_i q_j}{m_i}}{|\vec{A_i A_j}|^3} \cdot \vec{A_i A_j}}
\end{align*}

Nyní si zvolme jistou (dostatečně malou) časovou konstantu $ dt $
reprezentující přesnost simulace pohybu částic v prostoru. Zjistíme, že za
časový okamžik $ dt $ částice $ A_i $ změní svoji polohu a rychlost
následujícím způsobem:

\begin{align*}
A_i' &= A_i + \vec{v_i} dt + \frac{1}{2}\vec{a_i} dt^2 \\
\vec{v_i'} &= \vec{v_i} + \vec{a_i} dt
\end{align*}

Poslední tři uvedené rovnice tvoří teoretický základ našeho algoritmu.

\subsection{Sekvenční algoritmus}

\begin{lstlisting}
for (step = 0; step < s.n_steps; ++step) {
    for (unsigned int i = 0; i < s.n_particles; ++i) {
        float ax = 0.0f;
        float ay = 0.0f;

        for (unsigned int j = 0; j < s.n_particles; ++j) {
            float dx = x[j] - x[i];
            float dy = y[j] - y[i];
            float invr = 1.0f / std::sqrt(dx * dx + dy * dy + 0.5f);
            float coef = (m[j] - q[i] * q[j] / m[i]) * invr * invr * invr;

            ax += coef * dx;
            ay += coef * dy;
        }

        xn[i] = x[i] + vx[i] * dt + 0.5f * ax * dt * dt;
        yn[i] = y[i] + vy[i] * dt + 0.5f * ay * dt * dt;
        vx[i] += ax * dt;
        vy[i] += ay * dt;
    }
}
\end{lstlisting}

Z výpisu kódu můžeme vidět, že jádro programu není dlouhé ani složité. Jedná
se víceméně o přímý přepis matematických rovnic uvedených výše až na několik
implementačních detailů.

Na řádku 9 si můžete všimnout, že před výpočtem odmocniny je k vstupní
hodnotě připočtena konstanta \texttt{0.5f}. Jedná se o protiopatření
pro případ, že $ i = j $; bez konstanty bychom se snažili odmocnit
nulu a následující dělení by selhalo pro dělení nulou. Nepřesnost
zanesená touto konstantou je zanedbatelná a jedná se výkonnostně o
lepší řešení než tento případ explicitně ošetřovat.

Na 10. řádku je zase zřejmé, že raději než třikrát dělit vzdáleností částic
je lepší si spočítat její převrácenou hodnotu a tou třikrát násobit. Výhodnost
spočívá v pozorování, že je obvykle časově náročnější provádět dělení než
násobení.

Poslední věcí stojící za zmínku je časová složitost algoritmu, která
je z výpisu kódu zcela zjevně $ O(n^2) $.

\pagebreak
\section{Optimalizovaná verze}

\subsection{Shrnutí}
V krátkosti můžeme říci, že nejvýraznějšího navýšení výkonu jsme
dosáhli rozbalením \texttt{i}-cyklu, ruční vektorizací a použitím
kompilační volby \texttt{-Ofast}. Následují grafy dosažené výkonnosti
a detailní popis jednotlivých vyzkoušených optimalizačních technik.

\begin{figure}[h!]
    \centering
    \includegraphics[width=0.9\textwidth]{graph_o3.eps}
    \label{fig:1}
\end{figure}

\begin{figure}[h!]
    \centering
    \includegraphics[width=0.9\textwidth]{graph_ofast.eps}
    \label{fig:2}
\end{figure}

\subsection{Loop tiling}
\begin{lstlisting}
for (unsigned int i = 0; i < s.n_particles; i += TILED_STEP) {
    unsigned int i1_end = std::min(i + TILED_STEP, s.n_particles);
    for (unsigned int j = 0; j < s.n_particles; j += TILED_STEP) {
        unsigned int j1_end = std::min(j + TILED_STEP, s.n_particles);
        for (unsigned int i1 = i; i1 < i1_end; ++i1) {
            float ax = 0.0f;
            float ay = 0.0f;
            for (unsigned int j1 = j; j1 < j1_end; ++j1) {
                float dx = x[j1] - x[i1];
\end{lstlisting}

Úplně první optimalizační technikou co jsme vyzkoušeli byl loop
tiling. Pro konstantu \texttt{TILED\_STEP} jsme použili mnoho hodnot,
avšak shledali jsme, že použití hodnot vyšších než $ 1 $ vesměs výkon
snižuje. Toto pozorování je konzistentí s měřením naivní verze, ve
které jsme nezaznamenali žádný pokles výkonu s rostoucím $ n $
(tj. připsatelný snížené efektivitě cache; nejvyšší dosažená hodnota
byla $ n = 440000 $, pro vyšší již výpočetní čas přesahoval limitních
60 minut). Došli jsme tedy k závěru, že loop tiling je v našem případě
bezcenný a proto se ani nenachází ve výsledných grafech.

\vspace{10pt}
\subsection{Rozbalení cyklů}
\begin{lstlisting}
for (unsigned int i = 0; i < n_particles - 1; i += 2) {
    float ax_0 = 0.0f;
    float ax_1 = 0.0f;
    float ay_0 = 0.0f;
    float ay_1 = 0.0f;
    for (unsigned int j = 0; j < n_particles; ++j) {
        float dx_0 = x[j] - x[i + 0];
        float dx_1 = x[j] - x[i + 1];
\end{lstlisting}

\begin{lstlisting}
for (unsigned int i = 0; i < n_particles; ++i) {
    float ax = 0.0f;
    float ay = 0.0f;
    for (unsigned int j = 0; j < n_particles - 1; j += 2) {
        float dx_0 = x[j + 0] - x[i];
        float dx_1 = x[j + 1] - x[i];
\end{lstlisting}

Zkoušeli jsme rozbalit jak \texttt{i}-cyklus, tak \texttt{j}-cyklus, a
to jak samostatně každý po 2, 4 a 8 krocích, tak i oba společně s
krokem 2. Zde jsme se dočkali příjemného překvapení, neboť společně s
volbou \texttt{-Ofast} zde došlo -- zejména po rozbalení
\texttt{i}-cyklu -- k výraznému nárustu výkonu; kompilátor zde zřejmě
již dokázal části cyklu automaticky vektorizovat. Rozbalení
\texttt{j}-cyklu ani obou cyklů společně však žádný výraznější nárust
výkonu nepřineslo.

\vspace{10pt}
\subsection{Vektorizace}
\begin{lstlisting}
for (unsigned int i = 0; i < n_particles - 3; i += 4) {
    __v4sf ax = { 0.0f, 0.0f, 0.0f, 0.0f };
    __v4sf ay = ax;
    const __v4sf xi = *(__v4sf *)(x + i);
    const __v4sf yi = *(__v4sf *)(y + i);
    const __v4sf mi = *(__v4sf *)(m + i);
    const __v4sf qi = *(__v4sf *)(q + i);
    for (unsigned int j = 0; j < n_particles; ++j) {
        const __v4sf xj = { x[j], x[j], x[j], x[j] };
        const __v4sf yj = { y[j], y[j], y[j], y[j] };
        const __v4sf mj = { m[j], m[j], m[j], m[j] };
        const __v4sf qj = { q[j], q[j], q[j], q[j] };
        const __v4sf dx = xj - xi;
        const __v4sf dy = yj - yi;
\end{lstlisting}

Pro vektorizaci jsme použili datový typ \texttt{\_\_v4sf} reprezentující
4 čísla s plovoucí desetinnou čárkou v jednoduché přesnosti. Tato optimalizace
přinesla s kompilační volbou \texttt{-O3} velmi podstatný nárust výkonu; verze
uvedená výše je pro tuto volbu nejvýkonnější (dosahuje cca. 3,5 GFLOPS, zatímco
naivní verze cca. 1,1 GFLOPS).

Pro volbu \texttt{-Ofast} je situace ještě o něco zajímavější, neboť
2-krát rozbalený a automaticky vektorizovaný \texttt{i}-cyklus má
srovnatelný výkon s kódem výše. Zašli jsme tedy ještě dál a ručně
vektorizovaný \texttt{i}-cyklus jsme rozbalili po 2, 3 a 4 krocích. To
zapříčinilo další růst výkonu a jako nejvýkonnější se kupodivu ukázala
být verze s 3-krát rozbaleným ručně vektorizovaným \texttt{i}-cyklem,
která dosahuje cca. 11 GFLOPS (naivní verze cca. 1,45 GFLOPS).

\pagebreak
\section{Paralelizovaná verze}

\subsection{Shrnutí}
Z dosažených výsledků je zřejmé, že úloha je velmi vhodná pro
paralelizaci.  Vyzkoušeli jsme několik variant, jak výpočet
paralelizovat, ale rozdíly ve výkonnosti byly zanedbatelné (viz graf
níže). Další informace a grafy týkající se nejvhodnějšího způsobu
paralelizace lze nalézt na konci této sekce.

\begin{figure}[h!]
    \centering
    \includegraphics[width=0.9\textwidth]{graph_paropts_ofastpar.eps}
    \label{fig:3}
\end{figure}

\subsection{Varianty plánování}
\begin{lstlisting}
for (step = 0; step < n_steps; ++step) {
#   pragma omp parallel for schedule(dynamic) default(none) \
        firstprivate(x, y, xn, yn)
    for (unsigned int i = 0; i < n_particles; ++i) {
\end{lstlisting}

Nejprve jsme zkoušeli varianty plánování (\texttt{static},
\texttt{dynamic} a \texttt{guided}) pro nejvýkonnější sekvenční
řešení. Z grafu je zřejmé, že zvolená varianta plánování nemá valný
vliv na výkonnost, což odpovídá faktu, že každá iterace
\texttt{i}-cyklu představuj stejný díl práce. Dále jsme uvažovali
pouze statické plánování.

\vspace{10pt}
\subsection{Sdílení proměnných}
\begin{lstlisting}
for (step = 0; step < n_steps; ++step) {
#   pragma omp parallel for schedule(static)
    for (unsigned int i = 0; i < n_particles; ++i) {
\end{lstlisting}

V tomto testu jsme zkusili odstranit direktivu pro nesdílení
ukazatelových proměnných mezi vlákny, abychom zjistili její
vliv na výkon. Z grafu ale vyplývá, že není valný, což odpovídá
očekáváním; do těchto proměnných není nikdy zapisováno přímo, používají
se pouze k přístupu do (mezi vlákny disjunktní) paměti.

\vspace{10pt}
\subsection{Paralelizace obou cyklů}
\begin{lstlisting}
#pragma omp parallel for schedule(static) default(none) \
    firstprivate(i, x, y, xn, yn) reduction(+:ax_0,ax_1,ay_0,ay_1)
    for (unsigned int j = 0; j < n_particles; ++j) {
\end{lstlisting}

Pro paralelizaci \texttt{j}-cyklu jsme byli nuceni využít
nevektorizovanou verzi algoritmu, neboť povaha výpočtu vyžaduje, aby
na akumulační proměnné zrychlení částic byla aplikována redukce, avšak
tato funkcionalita funguje pouze na standardní datové typy (tj. ne na
vektory). Využili jsme nejvýkonnější verzi rozbalení \texttt{j}-cyklu
(krok 2). Dopad na výkonnost je spíše negativní, zarazil nás však náhlý
propad výkonu verze bez paralelizace \texttt{j}-cyklu.

\vspace{10pt}
\subsection{Nejvhodnější způsob paralelizace}
\begin{lstlisting}
for (step = 0; step < n_steps; ++step) {
#   pragma omp parallel for schedule(static) default(none) \
        firstprivate(x, y, xn, yn)
    for (unsigned int i = 0; i < n_particles; ++i) {
\end{lstlisting}

Nejvhodnějším způsobem paralelizace jsme shledali verzi se statickým
plánováním bez sdílených proměnných a paralelizace \texttt{j}-cyklu.
Takto paralelizovaný nejvýkonnější sekvenční algoritmus jsme podrobili
testu zrychlení s různými počty vláken. Graf má velmi příjemný lineární
průběh až do plných 12 vláken.

\begin{figure}[h!]
    \centering
    \includegraphics[width=0.9\textwidth]{graph_threads_ofastpar.eps}
    \label{fig:4}
\end{figure}

Na závěr jsme se rozhodli změřit výkonnost všech sekvenčních
simulátorů paralelizovaných tímto způsobem. Z grafu vyplývá téměř
lineární zrychlení všech vektorizovaných algoritmů. Zarážející je
významné zpomalení některých simulátorů s rostoucím $ n $, zatímco
jiné toto zpomalení nedemonstrují. Nejvýkonnější je opět 3-krát
rozbalená vektorizovaná verze, která při 12 vláknech dosahuje
cca. 125~GFLOPS (oproti sekvenční verzi s cca. 11~GFLOPS).

\begin{figure}[h!]
    \centering
    \includegraphics[width=0.9\textwidth]{graph_ofastpar.eps}
    \label{fig:5}
\end{figure}

\pagebreak
\section{Závěr}
V této práci jsme implementovali řadu simulátorů pro pohyb částic v 2D
prostoru. Během implementace jsme došli k poznání, že charakter výpočtu
je velmi vhodný jak pro vektorizaci, tak pro paralelizaci.

Naivní verze algoritmu s aktivovanou volbou \texttt{-Ofast} dosahovala
výsledku cca. 1,45 GFLOPS. Tento údaj jsme brali jako základní pro další
porovnávání výkonnosti.

Jako nejlépe optimalizovaná verze se ukázala být ručně vektorizovaná
verze s 3-krát rozbaleným \texttt{i}-cyklem. Tento způsob výpočtu
dosáhl s \texttt{-Ofast} v jednom vláknu cca. 11 GFLOPS, což je téměř
8-násobné zrychlení oproti naivní verzi.

Po vyzkoušení několika způsobů paralelizace jsme zjistili, že nejvyšší
výkon má opět vektorizovaná verze s 3-násobným \texttt{i}-cyklem, dosahující
125 GFLOPS ve 12 vláknech (cca. 11,5-násobné zrychlení oproti sekvenční verzi
a cca. 86-násobné zrychlení oproti naivní verzi).

\end{document}

